% !TEX TS-program = xelatex
% !TEX encoding = UTF-8 Unicode
%-------------------------
% Resume in Latex
% Author : Sourabh Bajaj
% License : MIT
%------------------------

\documentclass[letterpaper,12pt]{article}

\usepackage{latexsym}
\usepackage[empty]{fullpage}
\usepackage{titlesec}
\usepackage{marvosym}
\usepackage[usenames,dvipsnames]{color}
\usepackage{verbatim}
\usepackage{enumitem}
\usepackage[hidelinks]{hyperref}
\usepackage{fancyhdr}
\usepackage{dashrule}
\usepackage{ifthen}
\usepackage{xepersian}
\settextfont[Scale=1]{IRANSansFaNum}

\pagestyle{fancy}
\fancyhf{} % clear all header and footer fields
\fancyfoot{}
\renewcommand{\headrulewidth}{0pt}
\renewcommand{\footrulewidth}{0pt}

% Adjust margins
\addtolength{\oddsidemargin}{-0.5in}
\addtolength{\evensidemargin}{-0.5in}
\addtolength{\textwidth}{1in}
\addtolength{\topmargin}{-.5in}
\addtolength{\textheight}{1.0in}

\urlstyle{same}

\raggedbottom
\raggedleft
\setlength{\tabcolsep}{0in}

% Sections formatting
\titleformat{\section}{
  \vspace{-4pt}\scshape\raggedleft\Large
}{}{0em}{}[\color{black}\titlerule \vspace{-5pt}]

%-------------------------
% Custom commands
\newcommand{\resumeItem}[3][]{
  \item\small{
    \textbf{#2}{\ifthenelse{\equal{#3}{}}{}{:\\#3} \vspace{-2pt}}{\begin{flushleft}\lr{\href{#1}{#1}}\end{flushleft}}
  }
}

\newcommand{\resumeSubheading}[4]{
  \vspace{-1pt}\item
    \begin{tabular*}{0.97\textwidth}{r@{\extracolsep{\fill}}l}
      \textbf{#1} & #2 \\
      \textit{\small#3} & \textit{\small #4} \\
    \end{tabular*}\vspace{-5pt}
}

\renewcommand{\labelitemi}{$\bullet$}
\renewcommand{\labelitemii}{$\circ$}

\newcommand{\resumeSubItem}[2]{\resumeItem{#1}{#2}\vspace{-4pt}}

\newcommand{\resumeSubHeadingListStart}{\begin{itemize}[leftmargin=*]}
\newcommand{\resumeSubHeadingListEnd}{\end{itemize}}
\newcommand{\resumeItemListStart}{\begin{itemize}}
\newcommand{\resumeItemListEnd}{\end{itemize}\vspace{-5pt}}

%-------------------------------------------
%%%%%%  CV STARTS HERE  %%%%%%%%%%%%%%%%%%%%%%%%%%%%


\begin{document}

%----------HEADING-----------------
\begin{tabular*}{\textwidth}{r@{\extracolsep{\fill}}l}
  \textbf{\href{https://morteza-karimi.ir/}{\Large مرتضی کریمی}} & پست الکترونیک : \href{mailto:me@morteza-karimi.ir}{\lr{me@morteza-karimi.ir}}\\
  متولد: ۱۳۷۳/۲۷/۰۳ & وضعیت تاهل: متاهل\\
  \href{https://morteza-karimi.ir/}{\lr{https://morteza-karimi.ir/}} & تلفن همراه : \lr{+98-921-635-1266} \\
\end{tabular*}


%-----------EDUCATION-----------------
\section{تحصیلات}
  \resumeSubHeadingListStart
    \resumeSubheading
      {دانشگاه صنعتی شاهرود}{شاهرود، ایران}
      {کارشناسی ارشد هوش مصنوعی؛  معدل ۱۶/۴۶}{شهریور ۱۳۹۶ - تاکنون}
    \vspace{6pt}\\
    \small{\textbf{عنوان پایان‌نامه:} بازشناسی چهره با استفاده از شبکه عصبی کپسول}
    \resumeSubheading
      {دانشگاه خیام مشهد}{مشهد، ایران}
      {کارشناسی مهندسی نرم افزار؛  معدل ۱۵/۴۷}{شهریور ۱۳۹۱- تیر ۱۳۹۶}
  \resumeSubHeadingListEnd

%-----------Artificial Intelligence-----------------
\section{تجربیات هوش‌مصنوعی}
\resumeSubHeadingListStart
\item{
	\textbf{شمارش تعداد و اندازه قطرات سیال و طراحی سیستم پیشبین برای  محاسبه اندازه قطره خروجی}{: در این پروژه با استفاده از زبان برنامه‌نویسی پایتون تعداد  و اندازه قطرات خروجی در سیستم آزمایشگاهی تشخیص و محاسبه گردید سپس با استفاده از پایگاه داده بدست آمده مدل پیشبین شبکه عمیق این‌ سیستم طراحی گردید تا بتوان با استفاده از پارامتر‌های ورودی تعداد و اندازه قطرات را پیشبینی نمود.}
	\hfill
}
\item{
	\textbf{پیاده‌سازی پروژه جدا سازی قطرات  روغن و آب}{: در این پروژه با استفاده از برنامه‌نویسی متلب، جداسازی و شمارش قطرات به طور خودکار با استفاده از مباحث پردازش تصویر انجام گرفت.}
	\hfill
}
\item{
	\textbf{تشخیص چهره با استفاده از یادگیری عمیق}{: در این پروژه با استفاده از روش‌های موجود در یادگیری عمیق تشخیص چهره با دقت بالا طراحی گردید.}
}
\item{
	\textbf{تعقیب چهره در ویدئو}{: در این پروژه با استفاده از زبان برنامه‌نویسی پایتون در کتابخانه \lr{OpenCV} تعقیب کننده طراحی گردید تا چهره انتخابی را در تصویر تعقیب نمایید.}
}
\item{
	\textbf{تشخیص پلاک خودرو در تصویر}{: در این پروژه که با استفاده از زبان برنامه‌نویسی پایتون ایجاد گردید در تصویر ورودی به محل پلاک خودرو تشخیص داده می‌شود.}
}
\item{
	\textbf{بهبود کیفیت تصویر دارای نویز}{: در این پروژه با استفاده از برنامه‌نویسی نرم‌افزار متلب سعی شد بهبود کیفیت در تصویر دارای نویز انجام شود تا قابل استفاده برای پردازش‌های بعدی گردد.}
}
\resumeSubHeadingListEnd
%-----------EXPERIENCE-----------------
\section{تجربیات}
  \resumeSubHeadingListStart

    \resumeSubheading
      {\href{http://zagstudio.ir/}{استودیو طراحی زاگ}}
		      {سبزوار، ایران}		
      {برنامه‌نویس وب}		
      {شهریور ۱۳۹۶ - تاکنون}		
      \resumeItemListStart
      	\resumeItem[http://portal.kamaposter.com]{پیاده‌سازی پرتال شرکت کاما پوستر}{طراحی پرتال با استفاده از فریمورک \lr{Yii2}}
		\resumeItem[http://bimehasiakarimi.com]{پیاده‌سازی پرتال داخلی بیمه آسیا شعبه سبزوار}{طراحی پرتال با استفاده از فریمورک \lr{Yii2}}
		\resumeItem[http://booma.ir]{طراحی سایت املاک بوما}{طراحی سایت با استفاده از فریمورک \lr{Yii2}}
        \resumeItem[https://shokufa.org]{طراحی سایت جشنواره آنلاین موسیقی شکوفا}{طراحی سایت با استفاده از فریمورک \lr{Yii2}}
        \resumeItem[http://rabaze.com]{سایت رابازه}{همکاری به صورت موقت در پیشرفت پروژه، این پروژه با استفاده از فریمورک \lr{Laravel} انجام شده است.}
      \resumeItemListEnd

      \hdashrule{\fill}{1pt}{1pt}
      
    \resumeSubheading{دانشگاه خیام}{مشهد، ایران}{برنامه‌نویس وب}{مهر ۱۳۹۳ - تیر ۱۳۹۶}
      \resumeItemListStart
        \resumeItem[http://ce.khayyam.ac.ir]
       {سایت انجمن کامپیوتر دانشگاه خیام مشهد}
        {طراحی سایت با استفاده از ادغام \lr{Node.js Platform} و فریمورک \lr{Yii2}}
        \resumeItem
        {سایت امور فرهنگی دانشگاه خیام مشهد}
        {طراحی سایت با استفاده از فریمورک \lr{Yii2}}
        
      \resumeItemListEnd
      
      \hdashrule{\fill}{1pt}{1pt}
      
    \resumeSubheading{سایر پروژه‌ها}{سبزوار، ایران}{برنامه‌نویس وب}{فروردین ۱۳۹۲ - تاکنون}
      \resumeItemListStart
        \resumeItem
        [https://promac.ir]{سایت شرکت پروماک}{طراحی سایت با استفاده از فریمورک \lr{Yii2}}
        \resumeItem
        [https://github.com/mortezakarimi/gentelella-rtl]
        {راست‌چین سازی قالب رایگان \lr{Gentelella}}
        {}
        
        \resumeItem
        [https://valencia.ir]{سایت کانون هواداران والنسیا در ایران}{طراحی سایت با استفاده از فریمورک \lr{Yii}}
        \resumeItem
        [http://a-rastegari.org]{سایت انجمن خیریه رستگاری }{طراحی سایت با استفاده از فریمورک \lr{Yii}}
      \resumeItemListEnd
  \resumeSubHeadingListEnd


%-----------PROJECTS-----------------
%\section{پروژه‌ها}

%
%--------PROGRAMMING SKILLS------------
\section{مهارت‌های برنامه‌نویسی}
  \resumeSubHeadingListStart
    \item{
      \textbf{زبان‌ها}{:\begin{latin}PHP, Python, Javascript, SQL, Matlab, C++, HTML, CSS, SCSS, TypeScript\end{latin}}
      \hfill
      }
    \item{
      \textbf{فریمورک‌های وب}{:\begin{latin} Yii2, Laravel, Yii, Bootstrap, JQuery, Angular\end{latin}}
      \hfill
      }
     \item{
      \textbf{کتابخانه‌های پایتون}{:\begin{latin} Numpy, Keras, Pandas, OpenCV
\end{latin}}
	}
  \resumeSubHeadingListEnd
%-------------------------------------------

\end{document}
